\documentclass[12pt,a4paper]{article}
\usepackage[utf8]{inputenc}
\usepackage[spanish]{babel}
\usepackage{amsmath}
\usepackage{amsfonts}
\usepackage{amssymb}
\usepackage{graphicx}
\usepackage[left=2cm,right=2cm,top=2cm,bottom=2cm]{geometry}
\usepackage{float}
\usepackage{caption}
\usepackage{subcaption}
\usepackage{multirow}
\usepackage{pict2e}
\usepackage{diagbox}
\usepackage[table]{xcolor}
\author{Carlos Alonso Aznarán Laos}
\title{Ejercicio Tabla}

\begin{document}
\maketitle
\section{Tabla 1}
\begin{tabular}{|l|c|}
\hline País & Población \\
\hline Bolivia & 10.1 \\
\hline Perú & 29.4 \\
\hline
\end{tabular}
\section{Tabla 2}
\begin{table}[h]
\centering
\caption{Población de Bolivia y Perú en 2011 (en millones)}
\begin{tabular}{lc}
\hline País & Población \\
\hline Bolivia & 10.1 \\ Perú & 29.4\\
\hline \hline
\end{tabular}
\caption*{\scriptsize Fuente: Ined, 2011} \end{table}
\section{Tabla 3}
\begin{table}[h]
\centering
\caption{Lima: Población por distritos según a~no censal}
\begin{tabular}{|l|c|c|c|}
\hline \bf Distrito & \multicolumn{3}{c|}{\bf Censo} \\
\hline & 1981 & 1993 & 2007 \\
\hline Ate&&&\\
\hline Barranco &&&\\
\hline
\end{tabular}
\end{table}
\section{Tabla 4}
\begin{table}[h]
\centering
\caption{Población...}
\begin{tabular}{|l|l|l|} \hline &&\bf Población \\ \hline
\multirow {3}{1.5cm}{\bf Censo} & 1981 & \\ & 1993 & \\ & 2007 & \\
\hline
\end{tabular}
\end{table}
\section{Tabla 5}
\begin{tabular}{|l|c|c|}
\hline
\diagbox[width=10em]{Distrito} {A~no\\Censal}&1993 & 2007\\
\hline & & \\
\hline
\end{tabular}
\section{Tabla 6}
\pagebreak

\begin{table}
\caption{Tabla con texto horizontal}
\begin{tabular}{|c|c|c|}
\hline Texto 1 & \rotatebox{90}{Texto 2\,} & Texto 3 \\ \hline Dato1&Dato2&Dato3\\
\hline
\end{tabular}
\end{table}
\section{Tabla 7}
\begin{table}[h]
\centering
\parbox{5cm}{ \caption{ \centering {Selección de países de América del Sur-2011: Población (en millones)}}} \\ {\rowcolors{2}{cyan!80!blue!50}{cyan!70!white!40} \begin{tabular}{ l c}
\hline País& Población \\
\hline Bolivia & 10.1 \\ Chile & 17.3 \\ Ecuador & 14.7 \\ Perú & 29.4 \\
\hline \end{tabular} } \\[0.1cm] \parbox{5cm}{\hspace{0.5cm} {\footnotesize Fuente: Ined, 2011}}
\end{table}
\end{document}