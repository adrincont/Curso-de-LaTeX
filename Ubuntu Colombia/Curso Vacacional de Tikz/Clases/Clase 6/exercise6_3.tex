\documentclass{article}
\usepackage[utf8x]{inputenc}
\usepackage[spanish]{babel}
\spanishdatedel
\usepackage{graphicx,float}
\usepackage{tikz}
\usetikzlibrary{graphdrawing,graphs,babel,arrows,calc,trees}
\usegdlibrary{layered, circular, force}

\begin{document}

\begin{figure}[H]
\centering
\begin{tikzpicture}[>=stealth, thick, rotate =90, green!75!black]
\graph[spring layout, nodes = {circle, draw}, node distance = 2.5cm]{
	1 ->[line width = 2pt, red] {2, 4}, % de vértices
	2 -> {4,5},
	3 -> {1,6},
	4 -> {3,5,6,7},
	5 -> 7,
	7 -> 6
};
\begin{scope}{gray!75}
\node[below] at ($(1)!.5!(2)$) {2};
\node[left] at ($(2)!.5!(5)$) {10};
\end{scope}
\end{tikzpicture}
\caption{Grafo con la librería force}
\end{figure}
\end{document}
Grafo ponderado, tiene peso cada arista. Por ejemplo puede ser la probabilidad entre conexión de un grafo y otro.

Los números son nodos etiquetados.