\documentclass{standalone}
\usepackage[utf8x]{inputenc}
\usepackage[spanish]{babel}
\usepackage{tikz}
\usetikzlibrary{positioning,calc,arrows,fit,babel}
%La librería positiioning nos va a permitir ubicar.
\begin{document}
\begin{tikzpicture}
\tikzset{
	nodo/.style = {fill = cyan!75!black, text width = 2.5cm, align = center, draw, rectangle, rounded corners = 3pt, minimum size = 1.5cm},
	
	general/.style = {align = center, draw, rectangle, rounded corners = 3pt, minimum size = 1.5cm},
	
	nivel1/.style = {general, font = \bf, fill = red!50, thick},
	
	nivelSec/.style = {general, fill = red!35},
	
	connection/.style = { >=stealth, ->, thin},
	nivel4/.style = {general, minimum size = 2.5 cm, fill = cyan
	},
}
%minimum weight, width, height también son váliidos.
\node[nivel1] (jefe) at (0,0) {Gerente};
\node[nivel1a] (secretaria) [below right = of jefe] {Secretaria de Gerencia};%at (2.5, -1.5)
\node[general] (subgerente) [below = 2.5cm of jefe] {Subgerente};
\node[general] (asistente) [below = of subgerente] {Asistente de gerencia};
\node[general] (coordinador1) at ($ (asistente) + (2, -2.5)$) {Coordinador de comunicaciones de gerencia};
\node[general](coordinador2) at ($(asistente) + (-2, -2.5)$) {Coordinador de comunicaciones de gerencia};
\node[general](coordinador3) [left = of coordinador2] {Coordinador \\ comité \\ ambiental};
\node[general](coordinador4) [right = of coordinador1] {Coordinador comité de seguridad y salud en el trabajo};

\node[inner sep = 0pt] (P) at ($(asistente) + (0, -1.75)$) {};
\draw[connection] (jefe) |- (secretaria);
\draw[connection] (jefe) --(subgerente);
\draw[connection] (subgerente) -- (asistente);

\draw[connection] (jefe) .. controls +(left:5cm) and +(left:5cm) and +(left:5cm) .. (asistente);
\foreach \position in {coordinador1, coordinador2, coordinador3, coordinador4}{
\draw[connection] (P.south) -- (\position.north);
}
\end{tikzpicture}
\end{document}
%(nombre del nodo) {etiqueta del nodo}
% no deberia contener espacio
%solucionar ifenación o identacion
\parindent=3em
Líneas de Bezier
Si entra por horizontal debe entrar en vertical o viceversa.
Y asignando los puntos de Control con las líneas de Bezier.

Tarea: Diseñar organigrama o mapa conceptual.
Deben tener conectores y palabras de conexión
%Comunidad LaTeX