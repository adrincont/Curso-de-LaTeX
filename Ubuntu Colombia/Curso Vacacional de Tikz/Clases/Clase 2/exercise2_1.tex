%Primer ejercicio.
\documentclass{standalone}
\usepackage[utf8]{inputenc}
\usepackage[spanish]{babel}
\usepackage[T1]{fontenc}
\usepackage{PTSansNarrow}
\usepackage[usenames,dvipsnames,x11names,table,svgnames]{xcolor}
\usepackage{tikz}
\usetikzlibrary{calc,babel,through,intersections,backgrounds}
\begin{document}
\begin{tikzpicture}[scale=2]
\coordinate[label = left:{\textcolor{DarkBlue}{$A$}}] (A) at (0, 0);
\coordinate[label = right:{\textcolor{DarkBlue}{$B$}}] (B) at (1.15, .75);

\draw[blue] (A) -- (B);

\node (D) [name path = D, draw, circle through = (B), label = left: $D$]  at (A) {};
\node (E) [name path = E, draw, circle through = (A), label = right: $E$]  at (B) {};

\path[name intersections = {of = D and E}];

\coordinate[label = above:{\textcolor{DarkRed}{$C$}}] (C) at (intersection-1);

\coordinate[label = below:{\textcolor{DarkRed}{$D$}}] (D) at (intersection-2);

%Método 1:
%\draw[red] (A) -- (C);
%\draw[red] (B) -- (C);

%Método 2:
\draw[DarkCyan] (A) -- (C) -- (B) -- cycle;
\draw[DarkGreen] (A) -- (D) -- (B) -- cycle;

%Método 3:
%\fill[black, opacity = .5] (A) circle[radius =1pt];
%\fill[black, opacity = .5] (B) circle[radius =1pt];
%\fill[black, opacity = .5] (C) circle[radius =1pt];

%Método 4:
\foreach \point in {A, B, C}{
	\fill[black, opacity = .5] (\point) circle[radius = 1.5pt];
}
\foreach \point in {A, B, D}{
	\fill[black, opacity = .5] (\point) circle[radius = 1.5pt];
}

\begin{pgfonlayer}{background}
\fill[yellow] (A) -- (B) -- (C) -- cycle;
\end{pgfonlayer}

\begin{pgfonlayer}{background}
\fill[DarkMagenta] (A) -- (B) -- (D) -- cycle;
\end{pgfonlayer}

\node[below, text width = 10cm, align = justify] at (.5,-1.5){
\small\textbf{Proposición 1:}\par
\emph{Construir un \textcolor{orange}{triángulo equilátero} sobre un \textcolor{blue}{segmento de recta} dado.}
\par\vskip2em
Sea \textcolor{blue}{AB} un segmento de recta, contruyendo...
};
%\coordinate [...] (Z) at (0+0.1*rand,0+0.1*rand);
\end{tikzpicture}
\end{document}