\documentclass{oromion}
\usepackage{lipsum}
\usepackage{natbib} % Formato numérico y autor-año.
\usepackage[pagebackref = true]{hyperref}
\title{Cuasi terminación del curso}
\subtitle[Terminando las tareas para obtener veinte de nota]
\institution{Universidad del Perú}
\department{Facultad de Ciencias}
\author{Oromion}
\filiationauthor{Universidad del Perú}
\tutor{Daniel Camarena}
\filiationtutor{Departamento de Matemáticas}

\begin{document}
\maketitle

\section{Científicos peruanos}

\subsection{Roger Metzger}

En esta parte del documento se insertará una referencia \citep*[ver][pág. 250]{Metzger2000} 
\lipsum[1-4]

Citando con alias \defcitealias{Metzger2000}{Peru}

\citepalias{Metzger2000}
\subsection{Hernán Neciosup}

Versión $\ast$ de los comandos de citación. Y aquí otra referencia 
\citet{Beltran2017}

\lipsum[1-8]
\section{Científicos extranjeros}

\subsection{Clements}
Citando \citep{RAAB20121290,Raab2013}

\lipsum[1-5]

\citetext{Ver \textbf{foliaciones} en \citealp{Beltran2017} o también álgebra diferencial en \citealp{Risch1969a}}

\subsection{Risch}
\cite{Risch1969a}

\lipsum[1-10]
\bibliographystyle{agsm}
\bibliography{mybib}
\end{document}

¿Qué es AGSMY?

\citealp quita los paréntesis