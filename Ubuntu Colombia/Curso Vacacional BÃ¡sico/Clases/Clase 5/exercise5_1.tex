\documentclass{article}
\usepackage[utf8x]{inputenc}
\usepackage[spanish]{babel}
\spanishdatedel
\usepackage{graphicx}
\usepackage[leqno]{amsmath} % Numeración de ecuaciones a la izquierda, sino estará en la derecha.
\usepackage{lipsum}
\usepackage{hyperref}
\usepackage{vmargin}
\setpapersize{A4}
\setmargins{2.5cm}{1.5cm}{16.5cm}{23.42cm}{10pt}{1cm}{0pt}{1cm}

\usepackage[table]{xcolor}
% Colores rgb
\definecolor{signBlue}{rgb}{0, .25, .53}
\definecolor{emeraldGreen}{rgb}{0, .79, .34}
\definecolor{topaz}{rgb}{0, .6, .88}
\definecolor{indigoDye}{rgb}{.05, .31, .55}
\definecolor{indigo2}{rgb}{.13, .53, .41}
\definecolor{blueSpider}{rgb}{.15, .27, .43}
\definecolor{darkOliveGreen2}{rgb}{.74, .93, .41}
\definecolor{ochre}{rgb}{.8, .47, .13}
\definecolor{darkOrange}{rgb}{.8, .4, .0}
\definecolor{skyblue6}{rgb}{.2, .6, .8}
\definecolor{firebrick}{rgb}{.7, .13, .13}
\definecolor{blueice}{rgb}{.85, .96, .94}
\definecolor{lightcopper}{rgb}{.93, .76, .58}

% Colores HTML
\definecolor{structColor}{HTML}{201090} % 015793
\definecolor{barColor}{HTML}{202122} % 202122
\definecolor{backColor}{HTML}{8998B5} % 8998B5
\definecolor{titlesColor}{HTML}{008100} % FF6614
\definecolor{myBlue}{HTML}{027FDF}
\definecolor{myGray}{HTML}{181818}
\definecolor{myRed}{HTML}{AA3939}
\definecolor{myGreen}{HTML}{32AB77}
\definecolor{myOrange}{HTML}{FF6302}
\title{Nuestro título}
\author{Oromion}
\date{\today}

\renewcommand{\thesection}{\alph{section}}
\newcounter{obs}
\newcommand{\obs}{\stepcounter{obs}{\bf Observación \theobs:}}

\hypersetup{
colorlinks = true,
filecolor = emeraldGreen
}

\begin{document}

\maketitle
\renewcommand{\contentsname}{Tabla de contenido}
\tableofcontents

\section{Grafo ponderado} \label{ponderado}

\lipsum[1-2]

\obs{} Esta es una primera observación

\section{Grafo nulo} \label{nulo}

\lipsum[1-3]
Aquí nos vamos a referir a la sección \ref{ponderado} actual con número \pageref{ponderado}.

\begin{equation} \label{media}
	\bar{x}=\frac{1}{n}\sum_{i=1}^{n}a_i
\end{equation}

\noindent
La sección \thesection aparece en la página \thepage .

\obs{} Esta es una observación importante \LaTeX{} este es el comando

\subsection{Generación de contadores}

\lipsum[1]

\subsubsection{Referencias cruzadas}

\lipsum[3]

La ecuación de la media tiene enumeración \eqref{media} aparece en la página \pageref{media}.

Un enlace dinámico al vídeo de Ideas Audaces en el cual participa de nuestro profesor de la Escuela de Ciencia de la Computación utilizando \texttt{href}. \href{https://www.youtube.com/watch?v=5dWaszzLNoE}{Juan Espejo -- Ideas Audaces}.

Enlace a un documento externo \href{run:./exercise5_1.pdf}{archivo generado}.

\begin{align*}
x = a \tag{A1} \label{A1}\\
y = b \tag{A2} \label{A2}
\end{align*}

\lipsum[1-3]

Referencia a \ref{A2}.
\end{document}