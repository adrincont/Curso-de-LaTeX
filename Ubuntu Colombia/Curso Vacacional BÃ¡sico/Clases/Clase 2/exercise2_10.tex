\documentclass[10pt]{article}
\usepackage[utf8x]{inputenc}
\usepackage[spanish]{babel}
\spanishdatedel
\usepackage[demo]{graphicx}
\usepackage{lipsum}
\usepackage{lscape}
\usepackage{tikz}
\usepackage{fancyhdr}
%El paquete fancydr Nos permite configurar el contenido de los pies de ṕágina.
\lhead{Autor}
\rhead{pág. \thepage}
\chead{\includegraphics[scale=.01,height=.5cm]{demo}}
\lfoot{\thesection}
\cfoot{}
\rfoot{\tikz \draw[magenta, opacity = 50] (0,0) circle[radius = 5pt];}

\usepackage[lmargin = 2.5cm, rmargin = 1.5cm, top = 3cm, bottom =1cm, foot = 5mm, head =1.5cm, width = 10cm]{geometry}% includeheadfoot

\renewcommand{\headrulewidth}{0pt} %Se elimina la línea del encabezado.
\renewcommand{\footrulewidth}{.5pt}

\author{Oromion}
\title{Lección V2.2}
\begin{document}
\maketitle
\begin{titlepage}
\thispagestyle{empty}
\newgeometry{hmargin = {1.5cm}, vmargin = {2cm}, nomarginpar, ignorefoot, ignorehead}
\center
\Huge{Título del libro}\\
\Large{Subtítulo del libro}

\vspace{3.5cm}

\Large{Universidad}\\
\small{Departamento}

\vspace{5mm}

\includegraphics[scale=.15]{example-image-a}

\vspace{4cm}

\begin{minipage}{.4\textwidth}
	Autor: Nombre del autor\\
	\small{Título del autor}
\end{minipage}
\hfill
\begin{minipage}{.4\textwidth}
	\flushright
	Jurado: Nombre del jurado\\
	\small{Título del jurado}
\end{minipage}

\vspace{2cm}

\begin{tikzpicture}
\draw[fill = magenta, opacity = .5] (0, .5) circle[radius =1];
\draw[fill = cyan, opacity =.5](.5, 0) circle[radius =1];
\draw[fill = yellow, opacity =.5](-.5,0) circle[radius =1];
\end{tikzpicture}

\vspace{4cm}

\today
\end{titlepage}
\pagestyle{fancy}
\section{Capítulo uno}
\lipsum[1-12]
\section{Nueva sección}
\end{document}